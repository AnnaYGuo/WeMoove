\section{Results}
\label{sec:results}

\indent We had one of our teammates record physical therapy exercises and used those videos to compare with 
the Cornell videos. From the human eye, it looked as though our teammate's video was extremely accurate, and 
our algorithm supported this theory.  

\indent Given a pose, we are able to identify similar poses in different videos from a database that we 
curated. This encourages the patient to be more engaged with the exercises, since the exercises do not have to 
 from the medical Cornell database, but rather also from dance.

\subsection{User Customization Options}
\indent Our project offers the user some simple customization options so that it can be catered to individuals. 
These are the difficulty levels and the body part isolation features.

\indent The difficulty is separated into three levels: easy, medium, and hard. Each level has a specific threshold 
of accuracy that is used to score the patient's video: Easy had a threshold of 60, medium 45, and hard 30. The 
difference between the angles of the two videos has to be less than or equal to this threshold to be considered “perfect” for each level. 

\indent The project can also isolate specific body parts to score. The accuracy of the exercises takes the whole body into consideration, 
so some body parts that remain idle will raise the accuracy just by being there. This feature allows the patient to decide 
what they want to be scored, so that if there is a specific body part they want to isolate in their practicing, they are 
able to get scores of that accordingly. This feature scores the same way as normal but utilizes the body part extraction feature 
from OpenPose. This can be especially useful if the patient is receiving treatment for a specific body part. For example, if 
they injured their arm, they may want feedback specifically on their arm to see how they are doing.
