\begin{abstract}
For many individuals facing health hurdles, from common ailments to severe injuries, physical therapy 
represents a beneficial routine that provides a tangible path to recovery. Because adherence to prescribed 
recovery plans is critical, we present a novel application of computer vision techniques to supplement the 
traditional physical therapy experience. Whereas many recent attempts to integrate technology have involved 
passive data-logging or costly equipment, we aim to provide a cheap but interactive experience revolving 
around two key features: 1) offer quantitative feedback by scoring patients’ poses and movements against 
their exercise videos and 2) empower users to curate personalized therapy routines by filtering a database 
of physical therapy videos with specific poses. In providing a measure of progress and facilitating 
selection of progressive exercises, we hope to encourage adherence to therapy plans and foster accelerated 
healing.
% Physical therapy can be likened to a 'rite of passage' that many people go through
% because of how common it is. It provides a solution to the patient's problem, whether
% that be back pain or a broken leg, but patients must be able to complete their assigned
% recovery plans in order to fully heal. To optimize patient recovery, we propose a novel
% approach to navigating their recovery plans utilizing computer vision techniques. By
% breaking down instruction videos into vectorized representations, we can break down the
% patient's videos similarly and compare the two, providing feedback and showing the
% patient's progress over time, thus providing motivation in their journey.
\end{abstract}