\section{Conclusion}
\label{sec:conclusion}

\indent Through this project, we were able to successfully utilize computer vision 
techniques to create a program tailored to improve patient adherence to 
recovery plans in physical therapy. We consider this to be the beginning 
of a greater work with its applicability not just in physical therapy 
but also in the healthcare industry and beyond.

\subsection{Limitations}
\indent Although we believe our project proves great advancement, it did 
come with its fair share of limitations. As previously stated the camera 
angle between the videos must be the same, otherwise the scores will not 
accurately reflect their performance. This is due to the fact that our 
Gesture Detection algorithm is overly simple. 

\indent Another major limitation we faced in the creation of the project 
is the time bottleneck from OpenPose. Due to our restrictions with GPU 
access, we were forced to downscale the data we processed; either 
through resolution or duration. Downscaling the resolution proved to 
reduce the most time, however, it also produced questionable keypoints. 
Thus, for the purpose of testing, we were forced to process shorter 
videos at full resolution, in order to get the most accurate keypoints 
from OpenPose; this resulted in a vast time limitation as a 16-second 
video took 34 minutes to process.

\subsection{Implications}
\indent For future implementations of this project, we would focus on 
matching movements, generate difficulty levels for exercises, and expand 
the scope of the project to also analyze other physical activities. 

\indent Currently, our project can match poses, but we would like 
to further develop this to match movements too. This way, if a patient 
wanted to practice a specific motion, like raising the arm, they would 
be able to easily find exercises that would help them practice this. 

\indent We would also like to have our project be able to give a 
difficulty rating for exercises as it analyzes the poses. This would be 
so that those in the early stages of physical therapy can see which 
exercises are useful to them. 

\indent The final update would be to expand the project’s scope to 
other physical activities. Specifically, we would like to focus on 
dance. The project would be able to determine a dancer’s accuracy, and 
offer many practice dance movements from various genres that the 
dancer can practice with. Another implication could be martial arts, 
which would work in the same way that physical therapy and dance would, 
but be tailored for any unexpected specifications that martial arts 
requires. 
