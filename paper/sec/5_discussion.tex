\section{Discussion}
\label{sec:discussion}

\indent This project had some unexpected setbacks which we worked towards solving. 

\indent To identify poses, we needed to find the distances between the angles of frames, 
but we originally had some trouble with the euclidean distance formula. The base formula 
was too sensitive to noise in the movements of the subject, which meant that the algorithm 
was overfitted to the data. By flooring some of the angle gradients, we were able to account 
for noise. This made the algorithm more resistant to slight differences in position, camera 
angle, orientation of the subject, etc. 

\indent Initially, the algorithm returned low accuracy scores for two videos that seemed 
otherwise quite similar. In order to ensure that the algorithm was working correctly as 
intended, an example video was run against itself, with an expected and actual accuracy 
result of 100\%. This demonstrated that the algorithm was working but needed to use some 
sort of threshold or margin of error.

\indent We also conducted manual testing ourselves, making sure that the accuracy score 
from the algorithm made sense with the two input videos. For example, if the videos are 
quite similar, we expect a high accuracy score, but if they are vastly different, 
we expect a low accuracy score.

\indent We tested individuals of different heights to ensure that height did not create 
a large difference in terms of accuracy score. For example, if the individual in the 
instruction video is very tall, but the patient is very short and follows the video perfectly, 
the calculation of the angles should ensure that height is not a factor that would limit 
the accuracy score.
